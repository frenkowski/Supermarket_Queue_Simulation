\section{Supermarket Model}
% convenzione cassa è celletta di fianco
% misure celle
% step e tempo giornata

\subsection{Struttura}
La struttura del supermercato viene caricata tramite un apposito file \lstinline{txt}. 
Tale file è strutturato in modo tale da fornire al modello alcune informazioni essenziali.

La prima riga prevede la dimensione del supermercato.
In questo studio si è deciso di modellarne uno di medie dimensioni, con un'area totale pari a $32x30$.

Sulla seconda riga è presente un parametro dipendente dall'area, ossia la capienza massima del supermercato; nel nostro studio sono considerate $35$ persone come capacità massima. 
%Questo risulta essere particolarmente importante per limitare l'effetto \textit{"sciame"}, il quale si verifica, nella modalità Classic, quando più agenti provano contemporaneamente a cambiare cassa. Limitandone l'effetto, le code risultano più verosimili.
%Nel nostro studio, quest'ultimo parametro ha un valore pari a $10$ mentre la capienza massima è stata impostata a $35$.

La terza riga descrive la tipologia di coda implementata. 
Sono presenti infatti due tipi di coda, \textbf{Classic} e \textbf{
Snake}, come descritto nella Sezione \ref{}.

Infine, nelle restanti righe, è presente la struttura vera e propria della mappa. 
Il simbolo \texttt{'X'} corrisponde ad un ostacolo, la \texttt{'O'} alle celle percorribili da un agente. 
I numeri da \texttt{1} a \texttt{N} equivalgono alle casse, mentre le lettere da \texttt{A} a \texttt{J} corrispondono ai punti di spawn. 
Questi sono necessari in quanto la fase di shopping, che vede l'agente muoversi tra gli scaffali con l'obiettivo di acquistare un certo numero di prodotti, viene interpretata dal nostro modello come una \textit{black-box}. 
Pertanto risulta inevitabile predisporre, in concomitanza di ogni scaffale, un'apposita cella che permette all'agente di raggiungere una determinata cassa, simulando così l'ingresso nell'area casse una volta terminato lo shopping. 
La modalità Snake prevede inoltre due simboli aggiuntivi, ovvero la lettera \texttt{S} che indica l'ingresso per la coda e la lettera \texttt{Z} che indica la relativa uscita. 

La Figura \ref{} mostra il contenuto del file \lstinline{txt} usato per la modalità Classic. La Figura \ref{} mostra invece la struttura della mappa una volta caricata dal modello.

\subsection{Tipologie di Coda}
Il modello implementa due diverse tipologie di coda:
\begin{itemize}
    \item \textbf{Classic} rappresenta, per l'appunto, la modalità più classica; è quindi presente un numero di code pari al numero di casse aperte. 
    In questo progetto una fila può avere una lunghezza massima pari a $11$ celle. 
    Questo valore permette di studiare efficacemente i fenomeni di nostro interesse. 
    \item \textbf{Snake} rappresenta invece la così detta modalità 'a serpentone', dove è presente un unico ingresso, il quale permette di accedere alla sola fila presente fino a raggiungere l'uscita. 
    Una volta raggiunta l'agente verrà assegnato alla prima cassa libera.
\end{itemize}
E' importante sottolineare come nella modalità Classic la scelta della cassa ottimale è rimandata all'agente, mentre nella modalità Snake è l'ambiente a dover effettuare questa operazione.

\subsection{Floor Field}
Nella modalità Classic, la scelta della destinazione ottimale, ossia della cassa più vicina e con il minor numero di persone in coda, avviene grazie all'utilizzo dei \textit{floor field}.
Nel dettaglio, la nostra implementazione vede la generazione di un numero di floor field pari al numero di casse. 
Il corrispondente valore di ogni cella, $p$, viene calcolato rispetto alla cassa $q$ come una distanza Euclidea:
\begin{equation*}
d\left(p, q\right) = \sqrt {\sum _{i=1}^{n}  \left( q_{i}-p_{i}\right)^2 }.
\end{equation*}
In corrispondenza di un ostacolo la distanza risulta pari a $+ \infty$.

%L'agente, avendo una visione locale dell'ambiente coincidente al vicinato di Moore, sceglie quindi, per ogni cassa, la cella locale ottima. 
%Avendo ora la cella locale ottima per ogni floor field, è possibile sommare il numero attuale di persone in coda, in modo da tener conto anche del relativo affollamento.
%Delle possibili celle, l'agente seleziona quella di valore minimo, la quale identifica la cassa scelta come obiettivo.

%Il successivo movimento dell'agente verrà approfondito nella Sezione \ref{}.

\subsection{Step del Modello} 
Ad ogni step del modello, vengono eseguite, nell'ordine, le due operazioni dettagliate di seguito.
Nel mezzo di esse, viene eseguito un singolo step per ciascun agente e successivamente vengono computate le attività di Data Collection, le quali verranno approfondite nella Sezione \ref{}.

\subsubsection{Ingresso degli Agenti}
Per far sì che il modello rappresenti una situazione vicina alla realtà, si è scelto di modellare l'ingresso degli agenti in modo tale da avere una fase iniziale di graduale riempimento, fino al raggiungimento di un picco oltre il quale ci sarà un progressivo svuotamento.

Viene quindi generato, ad ogni step, un numero $p_1 \in [0;1]$, secondo la seguente equazione:
\begin{equation*}
p_1 = -\frac{\cos\left(\frac{t\pi}{720}\right)}{2}+\frac{1}{2} .
\end{equation*}
Successivamente viene generato un numero casuale $p_2 \in [0;1]$. 
Se $p_2 \leq p_1$ allora l'agente verrà effettivamente creato ed effettuerà l'ingresso nel supermercato.

La Figura \ref{} mostra il grafico di tale funzione. 
E' possibile notare come la probabilità relativa all'ingresso di un agente tende ad aumentare fino al raggiungimento di metà giornata, quando il numero di step sarà pari a $720$, garantendo così un'affluenza sempre maggiore. 
Superato questo valore tale probabilità tenderà a scendere, determinando così un sempre maggior deflusso delle persone presenti nel supermercato.

\subsubsection{Gestione delle Casse}
La gestione delle casse è stata progettata in modo da garantire un numero di casse aperte coerente sia con la specifica fase della giornata, sia con il numero di persone all'interno del supermercato.
Questo implica che durante la fase di riempimento le casse tenderanno ad aprire. 
Durante il picco sarà presente una situazione in cui tutte le casse saranno aperte, in modo tale da gestire efficacemente la fase di maggiore affluenza. 
Durante lo svuotamento queste tenderanno a chiudere, in maniera speculare, . 

E' importante sottolineare come questa gestione non è stata implementata in maniera simmetrica ma, bensì, cercando di mitigare possibili sovraffollamenti, anticipando le aperture delle casse e posticipandone le corrispettive chiusure. 
